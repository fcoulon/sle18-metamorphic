\begin{abstract}
Domain-Specific Languages (DSLs) manifest themselves in remarkably diverse shapes, ranging from internal DSLs embedded as a mere fluent API within a programming language, to external DSLs with dedicated syntax and tool support.
% engineered using one of many language workbenches.
%
Although different shapes have different pros and cons, combining them for a single language is problematic:~language designers usually commit to a particular shape early in the design process and it is hard to reconsider this choice later on.
%
%Theoretically, this diversity should enable language designers to combine the strengths of multiple language engineering technologies in the development of a single DSL and enable language users to manipulate their abstractions in the most appropriate shape for the task at hand (\eg~domain modeling vs. system integration).
%In practice, however, combining multiple language implementation techniques is problematic.
%Language designers usually commit to a particular technology (\eg~Spoofax, EMF, Racket) and a particular shape (\eg~internal, external, API) early in the design process and this choice can hardly be reconsidered later on.
%
In this \emph{new ideas} paper, we envision a language engineering approach enabling (i) language users to manipulate language constructs in the most appropriate shape according to the task at hand, and (ii) language designers to combine the strengths of different technologies for a single DSL. 
%support of multiple shapes for a single DSL to better fit the various tasks to be performed with it. \td{Meh}
%report on early experiments and lessons learned building an incremental synchronization mechanism for \emph{metamorphic DSLs}.
%
We report on early experiments and lessons learned building \prism, our prototype approach to this problem.
We illustrate its applicability in the engineering of a simple shape-diverse DSL implemented conjointly in Rascal, EMF, and Java.
%Metamorphic DSLs are languages whose different constituents (syntax, semantics, tools) can be implemented using various language engineering technologies and which can adapt their shape to a particular user or task.
%Taking as an example a simple metamorphic FSM language implemented simultaneously in three technological spaces (Rascal, EMF, a fluent API in Java), we show how the same FSM models can be manipulated indifferently under various shapes (a textual program in Rascal, a graphical projection in EMF, a Java AST).
%This opens up the possibility to, \eg animate an FSM model in EMF while it is executed by an interpreter written in Rascal.
%
We hope that our initial contribution will raise the awareness of the community regarding these notions, and that the challenges we identify encourage future research.
%We intend to raise the awareness of the community on these notions, identify some of the challenges we encountered along the way, and encourage future investigations.
\end{abstract}
